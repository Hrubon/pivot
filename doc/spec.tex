\input opmac
\typosize[14/18]
\activettchar|

\sec User Interface (UI)

\secc Drawing the Network Graph

The single most important feature of the UI is to display a network graph. This
graph consists of routers and networks, where edges connect routers to the
networks.

During our preliminary experiments with the UI, we have reached agreement on the
following:

\begitems

* Nodes of the graph need to be placed automatically. For large networks, not
having some sort of automatic layout support would be very impractical. We need
to provide the user with a reasonable network graph even when the tool is used
for the first time. At the same time, the tool should permit manual adjustments
of the resulting drawing.

* For drawing of the network graph, we want to rely on the |dot| suite of graph
drawing tools, using the |neato| spring-based layout engine. This is a mature
program which provides decent drawings for sparse graphs, albeit with some
limitations which motivate the rest of this section.

* Similar graphs often result in very different drawings under the spring
model. This behavior is easily triggered by adding a single node or edge to an
existing graph. This is at odds with TODO described above, making it hard to
track visually how the topology changed over time.

* Ideally, it would be possible to have positions of certain nodes calculated
automatically, while other nodes could be ``pinned'' at the user-provided
positions. It is possible to provide the desired location of a node in |neato|,
but it is nonetheless subject to further transformations of the drawing, hence
not useful in our case.

\enditems

To provide the automatic layout feature and to overcome the limitations
described above at the same time, we have formalized the relationship of the
|neato|-obtained layout and the user's manually provided layout in the following
way.

Network graph layout can be switched between automatic and manual positioning
mode upon user's discretion. In the automatic mode, the position of each node
and edge in the graph is calculated using |neato|. In the manual mode, nodes
can be moved around and only labels and edges will be placed automatically by
|neato|. The automatic mode is fully automatic, i.e. the positions given by the
user are disregarded completely.

The user can switch between manual and automatic mode upon their discretion.
Automatic mode is the default. When the users switches from automatic to manual
layout for the first time, each node defaults to its |neato|-calculated
position.  If a new node appears in the graph, then in the automatic layout, its
position will be calculated by |neato| as that of any other node (possibly
resulting in a very different drawing compared to the previous one). In the
manual mode, it will be placed in the upper left corner and the user will
determine the final position of the node in the manual layout by moving it
around as desired.

This approach has the following benefits:

\begitems

* The user starts with a reasonably drawn graph which they can adjust as their
wish, probably to reflect the physical topology of the network in question.  If
the topology of the network changes significantly, the user can switch back to
automatic layout.

* |neato| still takes care of placement of edges and labels, even in the manual
mode, taking the burden off the user.

\enditems

\sec __Old__

\begitems\style a
* Web UI is a terminal, too.
* HTTP/2 server push to update content periodically if the client supports it,
  legacy polling mechanism otherwise
* JavaScript-based drag `n' drop UI to rearrange the nodes with possibly other
  ergonomic features.
* Fundamental operation of the Web UI (e.g. display/export of network map)
  must operate correctly without JavaScript support.
* Overall goal: leverage modern web technologies to provide value, not to bother.
\enditems

\begitems\style a
* Objects representing nets and routers can be moved. The relocation of an object
  results in a request being sent to the server, the object is then affixed to
  this position on the map.
* Upon next request/on next push update the client receives the new map where
  the object in question has assumed its new position.
* No intelligence in the redrawing of the map on the client, that's Graphviz's
  job to do.
* If an object has not been moved, its position is calculated by Graphviz
  (the |neato| layout engine in fact). If it has been moved before, its position
  is fixed. Such an object can be ``unpinned'' from the UI and will assume its
  calculated position in the future.
\enditems

\bye
